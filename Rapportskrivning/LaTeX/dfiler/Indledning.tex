\chapter{Indledning}\label{kapInd}
Processen er vigtig, når man udviklinger systemer, da man med god projektstyring og planlægning kan imødekomme uforudsete hændelser og derved undgå store forsinkelser i projektarbejdet. Ved brug af risikovurdering af projektet, kan man prioritere opgaver efter hvad der er størst risiko for kan ødelægge projektarbejdet. I dette bachelorprojekt er der taget udgangspunkt i den agile arbejdsproces Scrum, hvori man udfører timebestemte opgaver (tasks) i et interval (sprint) der i dette projekt har varet fra 2 til 3 uger. Der er til planlægning af hvert sprint blevet brugt risikovurdering til at bestemme prioriteten af især softwareopgaverne, for at undgå at udviklingsprocessen sættes i hold. 
I processen har gruppens tre medlemmer haft ligeligt indflydelse på indholdet af systemet. Til udarbejdelse af bilag som kravspecifikation, accepttest, design mm. er der anvendt UML og SysML til modellering af systemet, da dette er universelt anderkendt.

Til belysning af brugernes mening om systemet, er der anvendt kvalitative og kvantitative
metoder. Et kvantitativt spørgeskema er benyttet til at undersøge potentielle patienters
holdninger til robotteknologi i forbindelse med ultralydsscanninger, mens et kvalitativt
interview med henholdsvis en radiograf og en overlæge har været med til at belyse de sundhedsfagliges
synsvinkel. En medicinsk godkendelse har været en del af udviklingsprocessen, da systemet
vil defineres som medicinsk udstyr. Til den medicinsk godkendelse, er der fulgt direktiver og tilhørende harmoniserede standarder. 



