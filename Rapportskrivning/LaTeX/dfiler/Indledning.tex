\chapter{Indledning}\label{kapInd}
Processen er vigtig, i udviklingen af systemer, da der med god projektstyring og planlægning kan imødekomme uforudsete hændelser og derved undgå store forsinkelser i projektarbejdet. Ved brug af risikohåndtering af projektet, kan opgaver prioriteres efter, hvad der er størst risiko for, der forstyrrer projektarbejdet. I dette bachelorprojekt, er der taget udgangspunkt i den agile arbejdsproces, Scrum. I Scrum udføres timebestemte opgaver (tasks) i et interval (sprint), der i dette projekt har varet fra 2 til 3 uger. Der er til planlægning af hvert sprint blevet brugt risikovurdering til at bestemme prioriteten af især softwareopgaverne, for at undgå at udviklingsprocessen blev sat i hold. 
I processen har gruppens tre medlemmer haft ligeligt indflydelse på indholdet af Automatisk Ultralydsscanner. Til udarbejdelse af bilag som kravspecifikation, accepttest, design mm. er der anvendt UML og SysML til modellering af systemet, da dette er universelt anerkendt.

Til belysning af brugernes mening om en automatisk ultralydsscanner, er der anvendt kvalitative og kvantitative metoder. Et kvantitativt spørgeskema er benyttet til at undersøge potentielle patienters
holdninger til robotteknologi i forbindelse med ultralydsscanninger, mens et kvalitativt
interview med henholdsvis en radiograf og en overlæge har været med til at belyse de sundhedsfagliges
synsvinkel. Litteratursøgning er benyttet til at undersøge omkostningseffektiviteten ved at anvende screeningsprogrammer, og der er foretaget en økonomisk analyse for at undersøge omkostningerne ved at indføre Automatisk Ultralydsscanner. Undersøgelse af proceduren for medicinsk godkendelse er lavet, da Automatisk Ultralydsscanner defineres som medicinsk udstyr. Til den medicinsk godkendelse, er der fulgt direktiver og tilhørende harmoniserede standarder. 



