\chapter{Projektgennemførelse}\label{kapitel_Projektgennemforelse}
I dette afsnit vil de redskaber gruppen har anvendt i projektgennemførslen bliver beskrevet. 

\section{Gruppedannelse}
Gruppen består af Mathias Siig Nørregaard, informations- og kommunikationsteknologistuderende (IKT), samt Marie Kirkegaard og Charlotte Søgaard Kristensen, som begge er sundhedsteknologistuderende (ST). 

Oprindeligt bestod gruppen kun af Mathias og Marie, som fik kontakt med hinanden gennem en fælles veninde fra sundhedsteknologiuddannelsen årgang 2012.

Da projektet blev defineret, blev det klart, at  projektet var omfattende nok til at inkludere et trejde gruppemedlem. Charlotte er studiekammerat med Marie, og syntes projektet lød spændende. Hendes kompetencer passede godt ind i gruppen, og derfor blev hun det tredje og sidste medlem i bachelorprojektgruppen. 

Gruppen blev ved projektstart enige om at finde en reviewgruppe til feedback af gruppens arbejde og omvendt. Reviewgruppen bestod af Jonas Bæch (ST) og Kathrine Duus Kinnerup (ST), med lektor Samuel Alberg Thrysøe som vejleder. Det fungerede godt at have en gruppe med en anden vejleder, da vejlederne ikke altid ligger vægt på de samme produkt og proces-områder, og derfor havde reviewgruppen nogle forslag, som kunne forbedre bachelorprojektet.  

\section{Samarbejdsaftale}
Ikke alle gruppemedlemmerne kendte hinanden ved projektstart, og det var derfor vigtigt at få lavet en udførlig samarbejdsaftale som dokumentation for gruppens beslutninger og aftaler, samt rettesnor, hvis der skulle opstå problemer i samarbejdet. 

Det har ikke været nødvendigt at finde samarbejdsaftalen frem til konflikthåndtering under projektforløbet, da der ingen problemer har været med samarbejdet. Samarbejdsaftalen er derfor mest blevet brugt til forventningsafstemning af projektarbejdet samt til at diskutere forventninger og ønsker imellem gruppens medlemmer. 

Samarbejdsaftalen har været et vigtigt redskab for gruppen, da den kan have været medvirkende til at problemerne i samarbejdet aldrig er opstået. 

Samarbejdesaftalen er vedlagt i bilag \ref{Samarbejdsaftale}. 

\section{Arbejdsfordeling}
Arbejdsfordelingen har fungeret godt, da gruppens medlemmer har haft hvert deres ansvarsområder, hvor arbejdsfordelingen har været opdelt efter kompetencer og interesse hos gruppens medlemmer, samt hvad gruppens medlemmer kunne tænke sig at arbejde med i fremtiden.  

Mathias har primært haft ansvaret for kodningen af software til systemet. Marie og Charlotte har stået for rapportskrivning, brugerundersøgelser, risikovurdering og udarbejdelse af de fleste bilag. Marie har haft ansvaret for udarbejdelsen af den medicinske godkendelse, mens Charlotte har haft ansvaret for den overordnede projektstyring. 

Kravspecifikationen, accepttesten og design af systemet er diskuteret og udarbejdet i fællesskab, så alle gruppens medlemmer var indforstået med systemets krav og design. 

Da bachelorprojektet er et slags forskningssprojekt, har gruppens medlemmer ind i mellem været udfordret, da mange af opgaverne i projektarbejder var nyt stof for alle. Her har gruppens medlemmerne været gode til at hjælpe og spare med hinanden, samt søge hjælp hos vejleder og andre sparringspartnere.  

Nedenstående tabel viser fordelingen af ansvarsområder i gruppen. 

\begin{table}[h]
\centering
\begin{tabular}{|l| p{0.15\textwidth}|}
\hline
\textbf{Ansvarsområde} &  \textbf{Ansvarlig} \\\hline
Kravspecifikation og Accepttest & Fælles \\\hline
Udviklingsdokument & MSN, CSK\\\hline
Brugerundersøgelse & MK, CSK \\\hline
Udvikling af software & MSN\\\hline
Medicinsk godkendelse & MK \\\hline
Overordnet projektstyring & CSK \\\hline
\end{tabular}
\caption{Ansvarsområder}
\end{table}

\section{Planlægning}
Ved projektstart blev der lavet en standard sekventiel tidsplan. (Se bilaget Tidsplan). Denne tidsplan blev dog hurtigt droppet, da det var umuligt at planlægge sig ud af de alle ufordringer, som implementeringen ville give. Gruppen valgte i stedet at bruge Scrum som projektstyringsværktøj. Scrum blev valgt, da mange faser og tasks i projektet var ukendte, og det var derfor vigtigt at benytte et agilt værktøj gennem projektperioden, hvor gruppen kunne vende tilbage til de enkelte faser og tasks. Den oprindelige tidsplan ville virke bedre for et velkendt system, der skulle implementeres. Gruppen påbegyndte derfor implementeringen hurtigere end tidsplanen ville have gjort. De overordnede faser fra tidsplanen, lavet ud fra v-modellen, blev stadig forsøgt at blive overholdt, så det passede med  reviewgruppens faser. Dette var for eksempel færdiggørelse af kravspecifikation og accepttest. 

\section{Projektledelse}
Der har ikke været en officiel projektleder i gruppen, da gruppen blev vurderet for lille til at have en Scrum-master. Vigtige beslutninger er derfor taget kollektiv. Senere i forløbet blev det dog klart, at én i gruppen, var nødt til at have et overordnet overblik over projektforløbet. Dette blev Charlotte, som: 
\begin{itemize}
\item lavede udkast til sprints  med input fra alle gruppens medlemmer
\item havde oversigt over hvilke elementer der manglede i projektet
\item havde ansvaret for burn-down charts
\end{itemize} 

\section{Projektadministration}
Git, med SourceTree som interface, er blevet anvendt til versionsstyring af projektets dokumenter og kildekode. Dette gjorde, at det var let at se ændringer, finde frem til en tidligere version og håndtere merging (fusionering af dokumenter).

Alt intern kommunikation i gruppen, foregik blot via Facebooks Messenger-funktion, gruppens medlemmer kommunikerede mest verbalt. Ekstern kommunikation med vejleder Michael Alrøe og andre personer, som har hjulpet med projektet er foregået over e-mail. Se bilag \ref{Mails} Mails.  

Hver dag er dagens arbejde skrevet i en logbog. Samtlige færdiggjorte og igangsatte tasks og gruppens aftaler er blevet skrevet heri. Logbogen har hjulpet gruppen til at kunne finde tilbage til tidligere aftaler, samt huske hvor en task var blevet sluppet. Se bilag \ref{Logbog} Logbog.  

Websiden Trello er anvendt som scrumboard. Hvert Trello-board er et sprint, hvor listerne indeholder Backlog, Ongoing, Stalled, Review og Done. Trello har givet et godt overblik over de enkelte sprints og gruppens medlemmer har hele tiden kunne følge med i hvilke tasks som var i gang, hvem som lavede dem, hvilke tasks der var gået i stå og hvilke tasks der var færdige. 
\newpage
\section{Udviklingsforløb af koden}
Udviklingen af koden er sket i flere 'steps' ift. risikoanalysen (se bilag \ref{Risikovurdering}) af projektet. Dette er også gjort for løbende at få et Minimum Viable Product af højere og højere kvalitet. 
Ud fra risikoanalysen blev der vurderet af følgende 'steps' var vigtigst:

\begin{table}[h]
\centering
\begin{tabular}{|l| p{0.2\textwidth}|}
\hline
\textbf{Step} &  \textbf{Vigtighed} \\\hline
	Kommunikation med UR10 & Meget vigtigt \\\hline
	3D scan output fra Kinect & Meget vigtigt\\\hline
	Find sti af positioner fra 3D scan & Vigtigt \\\hline
	3D output behandling & Vigtigt \\\hline
	Beregn rotationer af fundne positioner & Mindre vigtigt \\\hline
\end{tabular}
\caption{Steps til udvikling af produkt}
\end{table}

Vigtigheden skal forstås sådan at de mindre vigtige opgaver afhænger af de mere vigtige opgaver. 
Det vil sige at gruppen har prioriteret at udvikle de vigtige opgaver først, så det til de næste sprints har været nemmere at vurdere tiden der skulle afsættes for at gennemføre de næste steps.

\section{Udviklingsforløb}
Der var taget udgangspunkt i V-modellen til at nå alle faser i projektarbejdet. Faserene var kravspecifikation, accepttest, design, implementering og test, og de stemte overens med reviewgruppens, sådan at de vigtigste udkast af dokumenter som kravspecifikation og accepttest kunne blive reviewet samtidigt. V-modellen indeholdte deadlines for hvornår de forskellige fasers udkast skulle være færdiggjorte, men der var ingen tidsrammer for hvor lang tid hver task i fasen måtte tage. Derfor besluttede gruppen også at anvende Scrum til at få struktur og styring på bachelorprojektets arbejdsopgaver.

I bachelorprojektet blev der anvendt en modificeret udgave af Scrum, hvor kun delelementer er benyttet. Projektet blev udarbejdet af tre medlemmer, hvilket har betydet, at der ikke har været en Scrum Master, og alle medlemmer har haft ansvar for processen. Product Owner kommer tættest på at være Søren Pallesen fra Robotic Ultrasound, men grundet arbejdstider er Product Owner fravalgt i denne proces. Søren Pallesen har haft rollen som sparringspartner gennem udviklingsperioden, hvor der i alt har været tre møder med ham. Se bilag \ref{Eksterne moder} Eksterne møder. 

Prioriteringen af tasks til de forskellige sprints har taget udgangspunkt i risikovurderingen, der blev lavet i starten af projektperioden. Her blev det vurderet, hvad der kunne går galt samt konsekvensen af og sandsynligheden for, at det ville gå galt. De tasks med høj risikoprofil, blev prioriteret først i projektarbejdet for at undgå blokader i projektet. Se bilag \ref{Risikovurdering} Risikovurdering.  

Hjemmesiden Trello er anvendt til at holde styr på de forskellige tasks. Hvert sprint har sit eget board. Hvert board er delt op i lister; Backlog, Ongoing, Stalled, Review, Done. Ved start af hvert sprint, er de forskellige tasks skrevet ind i Backlog og timesat. Den samlede timebestemmelse til hvert sprint blev udregnet ved at se, hvor meget tid hvert medlem havde til rådighed udover tid til andre studierelaterede opgaver. 

Efter hvert endt forløb er sprintet evalueret og et nyt er blevet planlagt. I de første 4-5 sprints havde alle teammedlemmer undervisning, hvilket har præget, hvordan sprints blev planlagt. Det betød, at nogle tidsbestemmelser ikke var præcise, da det til tider var meget svært at vurdere hvor meget af gruppens tid der skulle afsættes til øvrige fag - specielt under eksamensperioden. Det gav først mening at tage burn-down charts i brug omkring sprint 5, da gruppen på det tidspunkt var blevet meget bedre til at estimere tasks.

Alle backloggens tasks var indskrevet i excel, og ved færdiggørelse af en task, blev datoen noteret, og dokumentet opdaterede automatisk burn-down grafen, så man kunne se, hvordan processen lå i forhold til den lineære kurve. Se bilag \ref{Timebestemt sprints} Timebestemt sprints. 

!!HUSK Milepæle

Scrum er kun anvendt frem til den 9. december, da gruppen besluttede, at det ikke gav mening at tidssætte gennemlæsning af de udviklede dokumenter. Den sidste uge blev brugt på gennemlæsning, ensretning og færdigudvikling af dokumentationen for projektet.  

\subsection{Evaluering af de enkelte sprints} 
Dette afsnit vil kort opsummere de vigtigste punkter fra hvert sprint. Det har generelt været svært at timelægge ting, som man ikke har erfaring med - dette vil være en naturlig forhindring i et udviklingsprojekt hvor der delvist er praktisk forskning inde over. Den fulde evaluering af Scrum kan ses i bilaget Evaluering af Scrum.

\subsubsection{Sprint 1} 
Sprintet var kort og blev primært brugt til klargøring af projektet og undersøgelse af projekts indhold og emne. Det var først midt i dette sprint, at Scrum blev indført, indtil da havde gruppen anvendt en sekventiel tidsplan og enkelte opgaver var ikke timebestemte.

\subsubsection{Sprint 2} 
Det blev forsøgt at fordele tiden på bedst mulig vis på de forskellige tasks. Dele af gruppen havde aldrig arbejdet på denne måde før, og derfor tog nogle af taskene meget længere tid at løse end beregnet, mens andre tog meget kortere tid. Det var en stor udfordring at få defineret tasks og taskene var derfor blevet defineret meget bredt, var svære at gøre helt færdige og mange af taskene endte derfor i kategorien ”stalled”. 

\subsubsection{Sprint 3} 
Tasks blev defineret med et målbart resultat, så det var muligt at afslutte dem. Der var dog stadig nogle tasks, der ikke var mulige at afslutte, og som måtte videre over i sprint 4. Det var stadig svært for gruppen at vurdere, hvor lang tid taskene tog. Nogle tasks tog meget længere tid end forventet; fx har kravspecifikation og accepttest-arbejdet har taget lang tid. I dette sprint blev der brugt lang tid på research til f.eks. Kinect, som ingen i gruppen havde arbejdet med før.
 
\subsubsection{Sprint 4} 
Dette sprint bar præg af, at gruppens medlemmer uforudset havde meget travkt med eksamen. Derfor blev sprintet forlænget undervejs, da sprintet i første omgang var planlagt til kun at være i to uger. Sprintet blev forsøgt planlagt med udgangspunkt i risikovurderingen. Gruppen havde stadig svært ved at gennemskue specielt implementeringstasks. Dette gjorde, at der stadig var nogle tasks, der ikke kunne afsluttes fuldstændigt. Sprintets tasks blev næsten alle sammen nået, dog manglede nogle tasks, da tiden var blevet sat for lavt. 

\subsubsection{Sprint 5} 
Sprint 5 var et kort sprint, som blev evalueret med et burn-down chart, hvilket gav et godt overblik over sprintets forløb og hvor meget gruppen havde nået. Der var stadig tasks som ikke blev nået, men det var tydeligt at gruppen var blevet bedre til at timelægge sprintet og definere målbare tasks.

\subsubsection{Sprint 6} 
Planlægningen af sprintet fungerede godt, og der blev på baggrund af evalueringen af sprint 5 fundet forbedringer til produktet. Timebestemmelserne var igen mere præcise end ved tidligere sprint. Dog var en task som Medicinsk Godkendelse, som var sat til 35 timer en uoverskuelig task, som burde have været delt i mindre bider. På baggrund af denne erfaring, blev det besluttet at en task max måtte fylde otte timer. Nogle planlagte tasks fra sprintet blev udskudt til sprint 7, pga. nedbrud af udviklingscomputeren.   

\subsubsection{Sprint 7} 
7. og sidste sprint, blev brugt på at få alle løse ender samlet. Det blev besluttet, at stoppe sprintet den 9. december, for at bruge den sidste uge på review af rapport og dokumentation. Derudover besluttede gruppen at lave kodestop den 2. december, for at have tid til at skrive dokumentationen. 

Tasks blev i dette sprint, tidsmæssigt, planlagt godt. Gruppen havde et opfølgende møde med Lars Bolvig, hvilket gav nogle rettelser til rapporten. Matematikken bag rotationen af Robotarm, tog længere tid end beregnet. Der er igennem sprintet fundet nye tasks, da sprint 7 var det sidste, er alle opståede task taget med i sprintet, gruppen har derfor arbejdet mere end planlagt. 
 
\section{Møder}
!!HUSK SKAL DET HER MED? VI HAR IKKE GJORT DET EFTER SCRUM-RETORIKKEN.
%Der er arbejdet med daglige Scrum møder, når det har været muligt. Møderne har været en kort mundtlig opsummering af hvert medlems planer for dagen, hvilke forhindringer der kunne opstå og andet relevant.%

Vejledermøder har været planlagt efter behov, hvor hvert møde typisk har haft en varighed af 1 time. Gruppen har sendt en dagsorden til vejleder inden hvert møde, og der er skrevet referat til hvert vejledermøde. Se bilag \ref{Vejledermode} Vejledermøder.
 
Der har i projektperioden været møder med CEO Søren Pallesen, radiolog og utralydsekspert Lars Bolvig samt lektor Samuel Alberg Thrysøe. Møder med Søren Pallesen har omhandlet brugen af ultralyd og det fremtidige perspektiv med systemet. Mødet med Samuel Thrysøe har omhandlet spørgsmål til, hvordan specifikke tests af systemet kunne udføres ved brug af et 3D printet bryst. Møderne med Lars Bolvig har omhandlet teknisk viden omkring utralydsundersøgelse. Se bilag \ref{Eksterne moder} Eksterne møder. 

\section{Konflikthåndtering}
Samarbejdsaftalen (bilag \ref{ Samarbejdsaftale} har været et redskab i forebyggelsen af konflikter. Der har i løbet af projektarbejdet ikke været nogle alvorlige konflikter. De uenigheder, der var opstået, blev løst gennem god kommunikation, samt forståelse og gensidig respekt for hinandens synspunkter. Uenighederne har typisk omhandlet hvordan en specifik task eller lignende skulle løses eller beskrives i rapporten. Hvis det har været nødvendigt at inddrage alle gruppens medlemmer, for at nå til enighed, var det flertallet der bestemte løsningen. Gruppemedlemmerne er altid nået frem til en løsning, som hele gruppen var indforstået med.

\section{Opnåede erfaringer}
I løbet af projektperioden har gruppen opnået erfaringer inden for agil udvikling, tværfagligt gruppearbejde, kravspecifikation, testspecificering samt dokumentering af et produkt.

Gruppen har opnået erfaring med Scrum som processtyring. Især timebestemmelsen og definering af hvert task, er gruppens medlemmer gradvist blevet bedre til i løbet af hvert sprint. Derudover har gruppen også opnået erfaring med at prioritere tasks til sprints, hvilket er gjort ved brug af risikoanalysen af projektet. Risikoanalysen har dog ikke kunne bruges i alle tilfælde, da den blev lavet før gruppen kendte til alle projektets tasks. Derfor er risikoanalysen mest blevet anvendt til prioriteringen af tasks i softwareudviklingen, da gruppen kunne prioritere hvilke funktionaliteter de fandt vigtigst for Automatisk Ultralydsscanner. 

Gruppen har også opnået erfaringer i formuleringen og defineringen tests til accepttest. Gruppen havde problemer med at få UC3 : Hovedscenarie testet. Der blev brugt lang tid på at udtænke en test, som kunne teste Robotarms evne til at bevæge sig i et bestemt bevægelsesmønster. Gruppen prøvede at teste bevægelsesmønsteret ved at teste med maling, tape og tucsh, samt beklæde testobjektet med ler for at kunne se Robotarms bevægelsesmønster. 

Derudover har alle gruppens medlemmer opnået vigtig erfaring med tværfagligt samarbejde med personer man aldrig har arbejder sammen med før, hvilket er meget virkelighedsnært i forhold til et kommende arbejde i erhvervslivet.

Nedenfor har hvert gruppemedlem beskrevet deres opnåede erfaringer med projektarbejdet. 

\subsection{Charlotte}
Charlotte har i bachelorforløbet fået et større kendskab til projektdokumentation, brugerundersøgelser og projektstyring. Charlotte er blevet bedre til LaTeX redigering i løbet af projektet, men hun har har specielt øget sit kendskab til arbejdet med Scrum som projektstyringsværktøj og burn-down chart til at følge processen. Projektstyringen har givet hende en forståelse for planlægning både på langt og kort sigt i et projekt. Det tværfaglige arbejde i forbindelse med bachelorprojektet har fungeret godt, og det har givet en god dynamik og forbedret løsninger til forskellige problemstillinger i projektarbejdet. Projektarbejdet har været en virkelighedsnær oplevelse, da man i erhvervslivet vil skulle arbejde på tværs af professioner. 

\subsection{Marie}
I bachelorprojektet har Marie lært meget af at lave medicinsk godkendelse af Automatisk Ultralydsscanner. Dette har givet hende et godt kendskab til medical device direktivet, samt tilhørende standarder. Derudover har det også givet hende en forståelse af, hvor omfattende en opgave det er, for virksomheder, at lave medicinsk godkendelse af medicinsk udstyr. Arbejdet med bachelorprojektet har samtidigt givet Marie et godt kendskab til anvendelsen af Scrum som projektstyringsredskab, samt erfaring med tværfagligt arbejde med både Mathias og vejleder Michael Alrøe, som ikke kommer fra sundhedsteknologiuddannelsen og derfor er kommet med andre syn på og tilgange til dokumentation af projektarbejdet. Arbejdet med bachelorprojektet har også givet Marie et indblik i, at udvikle for en kunde med specifikke ønsker til produktet. 

\subsection{Mathias}
Mathias har fået endnu bedre kendskab til projektarbejdets processer. Selv om han før har prøvet at arbejde med samtlige af projekt-elementerne (Scrum, kommunikation, tværfaglighed, kravspecifikation, dokumentering osv.) der skulle til for at gennemføre opgaven, føler han at han nu er blevet stærkere på disse områder. Rent teknisk har han opnået erfaring inden for 3D-model teori, som fx udregning af rotationsvektorer, rumtransformation, mesh beregninger m.m. Mathias fik også bedre kendskab til problematikken bag 3D kameraer og robotarme.

\section{Fremtidigt arbejde}
I fremtiden kunne det forbedre projektarbejde at gruppen blev endnu bedre til at definere de enkelte task i sprintene. Det kunne for eksempel gøres ved at beskrive tasken målbart med et formål, et omfang og et mål. Det samme gælder specificering af hvert sprint, så der var et formål med sprintet og en beskrivelse af, hvor langt gruppen forventede at være med delsystemet når sprintet var afsluttet.

Derudover kunne gruppen have delt tasks op, så hvert task f.eks max var otte timer lang. Gruppen kunne måske også have haft fordel af at udføre daglige scrum standup møder, da gruppemedlemmer nogle gange arbejdede på det samme uden at vide det. 

Det havde også været en fordel for gruppen at bruge en continuous integration testing server. Dette kunne gøres for at gennemføre kontinuerlig og automatisk unittesting, og fordi unittesting af samtlige tests kan tage lang tid.