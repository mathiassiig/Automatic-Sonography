\chapter{Samlet Konklusion}\label{SamletKonklusion}
Udviklingsprocessen har været vigtig for projektarbejdet, og gennem projektarbejdet er der anvendt forskellige redskaber til at opnå det bedste mulige resultat. 

Gruppen havde i første omgang i sinde at planlægge projektarbejde efter en sekventiel model, men besluttede hurtigt at dette ikke ville fungere, da gruppen i stedet ønskede at benytte en agil arbejdsmetode. Denne beslutning blev taget på baggrund af projektets mange ukendte variabler. Gruppen er godt tilfreds med at have benyttet elementer af Scrum som projektets udviklingsramme, og anvendelsen af de overordnede faser i V-modellen som rettesnor.

Elementerne af Scrum der blev anvendt blev valgt ud fra, hvad der ville give værdi for udviklingsprocessen. På grund af gruppens størrelse blev der fravalgt at have en Scrum-master. Opdeling i sprints med task-boards og oprettelsen af burn-down charts blev anvendt, hvilket har givet meget god mening for gruppen. Gennem processen er gruppen med tiden blevet bedre til at anvende Scrum. Ved sprint 5 begyndte gruppe at bruge burn-down charts. Dette gav en god indikator for hvor god eller dårlig estimeringerne i sprintet var - og dagligt har gruppen kunne sige noget om hvor godt den var med i sprintet. Trello som Scrumboard har fungeret godt, og det samme gælder brugen af git som versionsstyringsværktøj.

Noget, der ikke har fungeret helt perfekt for gruppen, var risikovurdering af projektarbejdet til at prioritere projektets opgaver. Risikovurderingen var svær at anvende, da projektet delvist har været et forskningsprojekt og projektets problemstillinger og arbejdsopgaver derfor ikke kunne defineres nemt på forhånd. Dog lykkedes det gruppen at anvende risikovurdering i et begrænset omfang til prioritering af nogle opgaver i softwareudviklingen, hvilket fungerede udmærket. 

Selve gruppesammensætningen har været rigtig god, da gruppens medlemmer har suppleret hinanden godt og arbejdsfordelingen har været meget naturligt fordelt alt efter det enkelte medlems kompetence- og interesseområde. Gruppen startede samarbejdet med at lave en samarbejdsaftale, som hovedsageligt blev brugt til forventningsafstemning, og den kan være årsag til, at der ikke er opstået nogle konflikter i gruppen. 
