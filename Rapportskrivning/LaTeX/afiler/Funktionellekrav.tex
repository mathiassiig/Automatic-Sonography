\chapter{Tests af funktionelle krav}\label{Funktionellekrav}

\section{Test: UC1: Hovedscenarie}
\textbf{Use Case Navn:} Start system \newline
\textbf{Forsøgsopstilling:}
\begin{itemize}
\item Computeren er tændt
\item Robotarm er tilsluttet
\item Robotarm står i Ikke-standard Positur
\end{itemize}  
\textbf{Aktører:}
\begin{itemize}
\item Testperson
\item Robotarm
\end{itemize}  

\begin{table}[htb]
\begin{tabularx}{\textwidth}{|p{0.5cm}|X|X|p{2cm}|}
\hline
\textbf{} & \textbf{Test} & \textbf{Visuelle resultat} &\textbf{Godkendt}
\\
\hline
1.1 &  Testperson trykker på 'AutoSonography.exe' på computerens skrivebord & Menuen 'Startup Menu' vises på GUI & \checkmark \\\hline
2.1 &  Testpersonen venter på, at Robotarm flytter sig og stopper i en ny position & Robotarm er stoppet & \checkmark  \\\hline
2.2 &  På Robotarms touchskærm trykker Testperson på 'Move'-fanen øverst på GUI'en & En ny skærm vises, hvor der er mulighed for at aflæse koordinater i højre side & \checkmark \\ \hline
2.3 &  I højre hjørne af Robotarms skærm trykker Testperson på 'Feature'-dropdownen, og der vælges 'Base' & Z-koordinatet i rammen 'TCP' under 'Feature'-rammen er shiftet cirka $+$400 mm og farverne på knapperne i venstre side er skiftet. & \checkmark \\ \hline
2.4 & Testperson aflæser hver tekstboks i rammen 'TCP' og noterer, at de matcher hvert koordinat angivet i 'Standard Positur' i bilaget 'Sætningsliste' & Med en margin på $ \pm 1$ mm er Robotarms nuværende koordinater tilsvarende Posituren 'Standard Positur's koordinater & \checkmark \\ \hline
\end{tabularx}
    \caption{Test: UC1: Hovedscenarie}
    \label{uc1_test_h_label}  
\end{table}
\newpage

%%%%%%%%%%%%%%%%%%%%%%%%%%%%%%%%%%%%%%%%%

\section{Test: UC2: Hovedscenarie}
\textbf{Use Case Navn:} 3D scan brystområde \newline
\textbf{Forsøgsopstilling:}
\begin{itemize}
\item PC Applikation  er startet
\item 3D kamera er tilsluttet
\item Testobjekt er placeret inden for afgrænsning
\end{itemize}  
\textbf{Aktører:}
\begin{itemize}
\item Testperson
\item Testobjekt
\item 3D kamera
\end{itemize}  

\begin{table}[htb]
\begin{tabularx}{\textwidth}{|p{0.5cm}|X|X|p{2cm}|}
\hline
\textbf{} & \textbf{Test} & \textbf{Visuelle resultat} &\textbf{Godkendt} \\\hline
1.1 & Testperson trykker på knappen [3D Scan] på GUI's 'Startup Menu' & Skærmbilledet skifter til menuen '3D Scan Menu' & \checkmark  \\\hline
2.1 & Testperson trykker på knappen [Scan] på GUI's '3D Scan Menu' & På GUI bliver knappen [OK] tilgængelig & \checkmark \\\hline
3.1 & Testperson tjekker, om der er et dybdebillede på menuen '3D Scan Menu' & Menuen '3D Scan Menu' viser noget andet end et hvidt billede & \checkmark \\\hline
4.1 & Testperson trykker på knappen [OK] i menuen '3D Scan Menu' & Menuen 'Startup Menu' vises på GUI. Knappen [Ultralydsscan] er tilgængelig & \checkmark \\\hline
\end{tabularx}
    \caption{Test: UC2: Hovedscenarie}
    \label{uc2_test_h_label}  
\end{table}
\newpage

%%%%%%%%%%%%%%%%%%%%%%%%%%%%%%%%%%%%%%%%%

\section{Test: UC2: Udvidelse: Juster 3D billedets skæring}
\textbf{Use Case Navn:} 3D scan brystområde \newline
\textbf{Forsøgsopstilling:}
\begin{itemize}
\item PC Applikation er startet
\item 3D kamera er tilsluttet
\item Testobjekt er placeret indenfor afgrænsning 
\end{itemize}  
\textbf{Aktører:}
\begin{itemize}
\item Testperson
\item Testobjekt
\item 3D kamera
\end{itemize}  

\begin{table}[htb]
\begin{tabularx}{\textwidth}{|p{0.7cm}|X|X|p{2cm}|}
\hline
\textbf{} & \textbf{Test} & \textbf{Visuelle resultat} &\textbf{Godkendt} \\\hline
1.1 & Testperson trykker på knappen [3D Scan] på GUI's 'Startup Menu' & Skærmbilledet skifter til menuen '3D Scan Menu' & \checkmark  \\\hline
A1.1 & Testperson ændrer Y min til -0,14 på GUI's track bars & GUI viser Y Min til at være 0,14 og Y max til 0,3  & \checkmark \\\hline
A2.1 & Testperson trykker på knappen [Scan] på GUI's '3D Scan Menu' & På GUI bliver knappen [OK] tilgængelig & \checkmark  \\\hline
A2.2 & Testperson tjekker, at der der er et afskåret brystområde på menuen '3D Scan Menu' &  '3D Scan Menu' viser et afskåret brystområde af Testobjekt & \checkmark \\\hline
A2.3 & Testperson trykker på knappen [OK] i menuen '3D Scan Menu' & Menuen 'Startup Menu' vises på GUI. Knappen [Ultralydsscan] er tilgængelig & \checkmark \\\hline
\end{tabularx}
    \caption{Test: UC2: Undtagelse: Juster 3D billedets skæring}
    \label{uc2_test_e_1_label}  
\end{table}
\newpage

%%%%%%%%%%%%%%%%%%%%%%%%%%%%%%%%%%%%%%%%%

\section{Test: UC2: Undtagelse: Scanning er ikke godkendt}
\textbf{Use Case Navn:} 3D scan brystområde \newline
\textbf{Forsøgsopstilling:}
\begin{itemize}
\item PC Applikation er startet
\item 3D kamera er tilsluttet
\item Testobjekt er placeret indenfor afgrænsning 
\end{itemize}  
\textbf{Aktører:}
\begin{itemize}
\item Testperson
\item Testobjekt
\item 3D kamera
\end{itemize}  

\begin{table}[htb]
\begin{tabularx}{\textwidth}{|p{0.7cm}|X|X|p{2cm}|}
\hline
\textbf{} & \textbf{Test} & \textbf{Visuelle resultat} &\textbf{Godkendt} \\\hline
1.1 & Testperson trykker på knappen [3D Scan] på GUI's 'Startup Menu' & Skærmbilledet skifter til menuen '3D Scan Menu' & \checkmark  \\\hline
2.1 & Testperson trykker på knappen [Scan] på GUI's '3D Scan Menu' & På GUI bliver knappen [OK] tilgængelig & \checkmark  \\\hline
3.1 & Testperson tjekker, om der er et dybdebillede på menuen '3D Scan Menu' & Menuen '3D Scan Menu' viser noget andet end et hvidt billede & \checkmark \\\hline
B1.1 & Testperson godkender ikke billedet, men afskærer brystområde på Testobjekt, hvorefter Testperson trykker på knappen [Scan] på GUI's '3D Scan Menu' indtil der kommer et afskåret brystområde & På GUI bliver knappen [OK] tilgængelig, og menuen '3D Scan Menu' viser et afskåret brystområde på Testobjekt & \checkmark \\\hline 
B1.2 & Testperson trykker på knappen [OK] i menuen '3D Scan Menu' & Menuen 'Startup Menu' vises på GUI. Knappen [Ultralydsscan] er tilgængelig & \checkmark \\\hline
\end{tabularx}
    \caption{Test: UC2: Undtagelse: Scaning er ikke godkendt}
    \label{uc2_test_e_2_label}  
\end{table}
\newpage


%%%%%%%%%%%%%%%%%%%%%%%%%%%%%%%%%%%%%%%%%

\section{Test: UC3: Hovedscenarie}
\textbf{Use Case Navn:} Ultralydsscan brystområde \newline
\textbf{Forsøgsopstilling:}
\begin{itemize}
\item PC Applikation er startet
\item Robotarm er tilsluttet
\item Testobjekt har ikke skiftet position siden udførslen af UC2: 3D scan brystområde
\item 3D scanning af Testobjekt er godkendt
\end{itemize}  
\textbf{Aktører:}
\begin{itemize}
\item Testperson
\item Testobjekt
\item Robotarm
\end{itemize}  

\begin{table}[htb]
\begin{tabularx}{\textwidth}{|p{0.5cm}|X|X|p{2cm}|}
\hline
\textbf{} & \textbf{Test} & \textbf{Visuelle resultat} &\textbf{Godkendt} \\\hline
1.1 & Testperson trykker på knappen [Ultralydsscan] på GUI's 'Startup Menu' & Skærmbilledet skifter til menuen 'Ultralydsscan Menu' & \checkmark  \\\hline
2.1 & Testperson observerer, om Robotarm roterer omkring Testobjekt i et specifikt bevægelsesmønster beskrevet i bilag \ref{Kravspecifikation}  Kravspecifikationen & Robotarm roterer rundt om Testobjekt i det specifikke bevægelsesmønster & - Se bilag \ref{BilagAccepttest} for yderligere forklaring  \\\hline
3.1 & Testperson gennemgår tests fra UC1: Start system i Tabel \ref{uc1_test_h_label} fra punkt 2.1 til punkt 2.4  & Visuelle resultater fra punkt 2.1 til punkt 2.4 i Tabel \ref{uc1_test_h_label} & \checkmark  \\ \hline
\end{tabularx}
    \caption{Test: UC3: Hovedscenarie} 
    \label{uc3_test_h_label}  
\end{table}
\newpage

%%%%%%%%%%%%%%%%%%%%%%%%%%%%%%%%%%%%%%%%%

\section{Test: UC3: Udvidelse: Operatør pauser scanning}
\textbf{Use Case Navn:} Ultralydsscan brystområde \newline
\textbf{Forsøgsopstilling:}
\begin{itemize}
\item PC Applikation er startet
\item Robotarm er tilsluttet
\item Testobjekt har ikke skiftet position siden UC2: 3D scan brystområde
\item 3D scanning af Testobjekt er godkendt
\item Stopur
\end{itemize}  
\textbf{Aktører:}
\begin{itemize}
\item Testperson
\item Testobjekt
\item Robotarm
\end{itemize}  

\begin{table}[htb]
\begin{tabularx}{\textwidth}{|p{0.7cm}|X|X|p{2cm}|}
\hline
\textbf{} & \textbf{Test} & \textbf{Visuelle resultat} &\textbf{Godkendt} \\ \hline
1.1 & Testperson trykker på knappen [Ultralydsscan] på GUI's 'Startup Menu' & Skærmbilledet skifter til menuen 'Ultralydsscan Menu' & \checkmark \\ \hline
A1.1 & Testperson starter stopur og trykker på knappen [Pause], mens Robotarm roterer rundt på Testobjek & Stopur er startet, og Robotarm stopper med at flytte sig på Testobjekt &  \checkmark \\\hline
A1.2 & Testperson slukker for stopur, når Robotarm stopper & Stopur viser 7 sekunder eller derunder &  \checkmark \\\hline
A2.1 & Testperson starter stopur & Stopur er startet &  \checkmark \\\hline
A2.2 & Testperson trykker på knappen [Resume] og stopper stopur, når Robotarm bevæger sig & Robotarm genoptager scanning af Testobjekt, og stopur viser 2 sekunder eller derunder &  \checkmark \\\hline 
2.1 & Testperson gennemgår tests fra hovedforløbet i UC2: Ultralydsscan brystområde i Tabel \ref{uc3_test_h_label} fra punkt 2.1 til punkt 3.1  & Visuelle resultater fra punkt 2.1 til punkt 3.1 i Tabel \ref{uc1_test_h_label} &  \checkmark \\ \hline
\end{tabularx}
    \caption{Test: UC3: Udvidelse: Operatør pauser scanning} 
    \label{uc3_test_e_1_label}  
\end{table}
\newpage

%%%%%%%%%%%%%%%%%%%%%%%%%%%%%%%%%%%%%%%%%

\section{Test: UC3: Undtagelse: Operatør stopper scanning}
\textbf{Use Case Navn:} Ultralydsscan brystområde \newline
\textbf{Forsøgsopstilling:}
\begin{itemize}
\item PC Applikation er startet
\item Robotarm er tilsluttet
\item Testobjekt har ikke skiftet position siden UC2: 3D scan brystområde
\item 3D scanning af Testobjekt er godkendt
\end{itemize}  
\textbf{Aktører:}
\begin{itemize}
\item Testperson
\item Testobjekt
\item Robotarm
\end{itemize}  

\begin{table}[htb]
\begin{tabularx}{\textwidth}{|p{0.7cm}|X|X|p{2cm}|}
\hline
\textbf{} & \textbf{Test} & \textbf{Visuelle resultat} &\textbf{Godkendt} \\ \hline
1.1 & Testperson trykker på knappen [Ultralydsscan] på GUI's 'Startup Menu' & Skærmbilledet skifter til menuen 'Ultralydsscan Menu' &  \checkmark \\\hline
B1.1 & Testperson trykker på knappen [Stop], mens Robotarm roterer rundt om Testobjekt & Robotarm stopper med at flytte sig på Testobjekt  &  \checkmark\\\hline
B2.1 &  Testperson gennemgår tests fra UC1: Start system i Tabel \ref{uc1_test_h_label} fra punkt 2.1 til punkt 2.4  & Visuelle resultater fra punkt 2.1 til punkt 2.4 i Tabel \ref{uc1_test_h_label} &  \checkmark \\\hline
\end{tabularx}
    \caption{Test: UC3: Undtagelse: Operatør stopper scanning} 
    \label{uc3_test_e_2_label}  
\end{table}
\newpage

%%%%%%%%%%%%%%%%%%%%%%%%%%%%%%%%%%%%%%%%%
\section{Test: UC4: Hovedscenarie}
\textbf{Use Case Navn:} Stop system \newline
\textbf{Forsøgsopstilling:}
\begin{itemize}
\item System er startet
\item Robotarm er tilsluttet
\end{itemize}  
\textbf{Aktører:}
\begin{itemize}
\item Testperson
\item Robotarm
\end{itemize}  

\begin{table}[htb]
\begin{tabularx}{\textwidth}{|p{0.5cm}|X|X|p{2cm}|}
\hline
\textbf{} & \textbf{Test} & \textbf{Visuelle resultat} &\textbf{Godkendt} \\ \hline
1.1 & Testperson trykker på knappen [Luk Knap] i GUI's øverste højre hjørne & Vinduet lukker ned &  \checkmark \\ \hline
2.1 & Testperson gennemgår tests fra UC1: Start System i Tabel \ref{uc1_test_h_label} fra punkt 2.1 til punkt 2.4  & Visuelle resultater fra punkt 2.1 til punkt 2.4 i Tabel \ref{uc1_test_h_label} &  \checkmark \\ \hline
\end{tabularx}
    \caption{Test: UC4: Hovedscenarie}
    \label{uc4_test_h_label}  
\end{table}
