\chapter{Ikke-funktionelle krav}\label{Ikkefunktionellekrav}

Ikke-funktionelle krav er begrænsninger til løsning af projektets funktionelle krav. Til at beskrive de ikke-funktionelle krav er MoSCoW og FURPS+ metoderne anvendt. 
MoSCoW-metoden betegner krav, som systemet skal opfylde (must), de krav som systemet bør realisere (should), de krav som systemet kunne realisere, men ikke har indvirkning på de andre krav (could), og de krav som omhandler fremtidige opdateringer og udvidelser eller krav som ikke implementeres  (would/won't).
  
FURPS+ står for:
\begin{enumerate}
\item[F.] Funktionelle krav, som er angivet i Use Cases.
\item[U.] Usability
\item[R.] Reliability
\item[P.] Performance 
\item[S.] Supportability 
\item[+.] Ekstra krav til systemet, som ikke hører ind under ovenstående. 
\end{enumerate}

%%%
\section{Usability}
\begin{enumerate}
    \item[U1.] PC Applikation skal have en GUI. (must)
    \item[U2.] Operatør bør kunne lære at ultralydsscanne Patient med Automatisk Ultralydsscanner på 2 timer. (should)
    \item[U3.] Systemet bør have en brugervejledning.  (should)  
    \item[U4.] Operatør, med kendskab til ultralyd, bør kunne betjene PC Applikation. (should)
    \item[U5.] PC Applikation bør kunne hente gamle målinger. (could)
\end{enumerate}

%%%
\section{Performance}
\begin{enumerate}
    \item[P1.] Scanning med 3D kamera og ultralydsscanning skal max tage 10 minutter til sammen. (must) 
    \item[P2.] Starttid på PC Applikation skal være max 10 sekunder. (must)
    \item[P3.] 3D kamera skal max bruge 10 sekunder på at tage 3D billedet. (must)
    \item[P4.] PC Applikation skal max bruge 10 sekunder på at færdiggøre brystområdets positurer til Robotarm. (must)
    \item[P5.] Robotarm bør have en tryksensor monteret. (should)
\end{enumerate}

%%%
\section{Supportability}
\begin{enumerate}
    \item[S1.] Ultralydsscanners probe bør kunne desinficeres med hospitalssprit. (should)
    \item[S2.] Operatør bør have mulighed for at skifte Ultralydsscanner til systemet. (should)
    \item[S3.] PC Applikation bør benytte n-tier architecture. (should)
    \item[S4.] Softwaren bør opbygges efter programmeringsprincipperne SOLID. (should)
\end{enumerate}
%%%
\section{Ekstra(+)}\label{andrePlus}

\begin{enumerate}
	\item[+1.] Systemet bør overholde Medical Device Directive 93/42/EØF \cite{MDD} (should)
	\item[+2.] PC Applikation bør overholde Standarden DS/EN 63204:2006 - Software for medicinsk udstyr - Livscyklusprocesser for software \cite{software} (should)
    \item[+3.] I fremtiden kan PC Applikation opdateres med en funktion, der lokaliserer og identificerer knuder. (would)
    \item[+4.] I fremtiden kan Operatør registrere en Patient i PC Applikation. (would) 
    \item[+5.] I fremtiden kan Operatør gemme og hente en måling i PC Applikation. (would)  
    \item[+6.] I fremtiden kan Operatør slette en måling i PC Applikation. (would)
    \item[+7.] I fremtiden kan PC Applikation have en database til at lagre data. (would)
    \item[+8.] I fremtiden kan Operatør tilgå PC Applikation gennem login. (would) 
    \item[+9.] I fremtiden kan Operatør identificere Patient i PC Applikation ud fra et patientidentifikation. (would)
\end{enumerate}
