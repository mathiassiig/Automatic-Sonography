\chapter{Konklusion}\label{kapitel_Konklusion}
Automatisk Ultralydsscanner er i stand til at detektere et brystområde med 3D kamera og derefter bevæge Robotarm rundt på det detekterede område, Robotarms yderste led roteres mod fladen af det detekterede område. I Automatisk Ultralysscanners nuværende version er den ikke i stand til at udføre fulde ultralydsscanninger af brystområder. For at en automatisk ultralydsscanning vil kunne udføres, med Automatisk Ultralydsscanner, vil det være nødvendigt at Robotarm kan bevæges i et specifikt bevægelsesmønster over brystområdet, samt at de øvrige implementeringer, nævnt i afsnittet om fremtidig udvikling, er implementeret. Det kan konkluderes at det er realistisk at kunne implementere en automatisk ultralydsscanner. 

Omkostninger til screeningsprogrammet ved tilføjelse af ultralydsscanninger vil øges. Prisen per scanning er billigere ved brug af Automatisk Ultralydsscanner, end hvis en radiolog har transporttid for at udføre scanningerne. Der vil være omkostninger til indkøb, opsætning og oplæring ved indførslen af Automatisk Ultralydsscanner.  

Det er svært at drage konklusion om, hvorvidt en udvidelse af screeningsforløbet vil være en god idé. Litteratur på området belyser, at mere forskning er påkrævet området, før men kan sige noget om dette. Fordelen ved udvidelsen er, at man ved en kombination af ultralyd og røntgen kan opdage tidligere stadier af kræft, og at overlevelsesprocenten  derved er højere. Ulemperne er, at der sker overdiagnosticeringer, og patienter derfor behandles uden grund.  Det er sandsynligvis ikke omkostningseffektivt, at tilføje ultralydsscanninger til screeningsprogrammet. 

Inden Automatisk Ultralydsscanner kan CE-mærkes, skal MDD overholdes. Det vil kræve udarbejdelse af teknisk dokumentation for produktet, bestående af en risikoanalyse og klinisk evaluering. Der vil skulle laves et kvalitetssikringssystem og producenten skal have et post market surveillance system. Derudover skal der vælges et bemyndiget organ, som godkender, at dokumentationen lever op til gældende lovgivning. Disse krav er ikke blevet implementeret i Automatisk Ultralydsscanner, da der var været mest fokus på udviklingsprocessen. 