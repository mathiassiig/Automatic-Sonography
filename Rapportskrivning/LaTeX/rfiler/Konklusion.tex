\chapter{Konklusion}\label{kapitel_Konklusion}
Automatiseret Ultralydsscanner er i stand til at detektere et brystområde med 3D kamera og derefter bevæge Robotarm rundt på området, men systemet er i sin nuværende form ikke i stand til at lave professionelle scanninger for brystkræft. En radiolog, der scanner for brystkræft vil føre ultralydsproben i et specifikt bevægelsesmønster over hele brystet og i armhulen, hvilket Automatisk Ultralydsscanner ikke er i stand til. Automatisk Ultralydsscanner er begrænset til kun at kunne scanne et detekteret område, som er scannet af 3D Kamera monteret i loftet, hvorfor hele brystets form og armhule placering ikke bliver detekteret. Automatisk Ultralydsscanner kan ikke registrere, hvor hårdt der trykkes på patienten, hvilket en automatiseret ultralydsscanner mhp. på screening for brystkræft bør have ift. patientsikkerhed. 

En automatiseret ultralydsscanning kan gavne økonomisk, hvis der allokeres opgaver fra radiologer til radiografer, da der er stor forskel på timelønnen. Den automatiserede ultralydsscanning af mamma kan foretages af radiografer, hvorefter radiologen vil sidde og vurdere scanningerne, ligesom proceduren er ved mammografiscreeninger. Denne model vil kunne spare transporttid for radiologen, og prisen pr. ultralydsscanning vil dermed blive lavere.  

Inden Automatisk Ultralydsscanner kan CE-mærkes, skal producenten overholde væsentlige krav fra DMD. Der skal udarbejdes teknisk dokumentation for produktet, bestående af en risikoanalyse og klinisk evaluering. Der vil skulle laves et kvalitetssikringssystem og have post market surveillance system. Derudover skal der vælge et bemyndiget organ, som godkender, at dokumentationen lever op til gældende lovgivning. Disse krav er ikke blevet implementeret i Automatisk Ultralydsscanner, da kendskabet til MDD først kom efter udviklingsprocessen var startet.

På Aarhus Universitetshospital, Tage-Hansens Gade, var man positive overfor en automatiserede ultralydssanning, da det kan give økonomisk mening og afhjælpe problemet med manglen på radiologer. Der vil være risiko for, at radiologen vil bruge mere tid på at analysere og vurdere billederne. 
De potentielle patienter fra spørgeskemaundersøgelsen var generelt positive overfor et system, og de fremhævede vigtigheden ved robottens følsomhed overfor tryk. Der blev det formodet , at en automatiseret ultralydsscanning vil kunne give økonomisk mening med kortere ventetider, spare tid og dermed frigørelse af ressourcer i form af personale til andre opgaver. 
