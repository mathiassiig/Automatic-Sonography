\chapter{Konklusion}\label{kapitel_Konklusion}
Automatisk Ultralydsscanner er i stand til at detektere et brystområde med 3D kamera og derefter bevæge Robotarm rundt på det detekterede område, hvor dens yderste led roteres nogenlunde mod fladen af området. I Automatisk Ultralysscanners nuværende version er den ikke i stand til at udføre fulde ultralydsscanninger af brystområder. For at en automatisk ultralydsscanning vil kunne udføres med Automatisk Ultralydsscanner, vil det være nødvendigt at Robotarm kan bevæges i et specifikt bevægelsesmønster over brystområdet, samt de øvrige implementeringer nævnt i afsnittet om fremtidig udvikling. Det kan konkluderes at det er realistisk at kunne implementere en automatisk ultralydsscanner. 

Automatisk Ultralydsscanner kan gavne økonomisk, hvis der allokeres opgaver fra radiologer til radiografer, da der er stor forskel på timelønnen. Automatisk Ultralydsscanner kan anvendes af radiografer, hvorefter radiologen vil skulle sidde og vurdere scanningerne, ligesom proceduren er ved mammografi. Denne model vil kunne spare transporttid for radiologen, og prisen pr. ultralydsscanning vil dermed blive lavere.  

Inden Automatisk Ultralydsscanner kan CE-mærkes, skal MDD overholdes. Der skal udarbejdes teknisk dokumentation for produktet, bestående af en risikoanalyse og klinisk evaluering. Der vil skulle laves et kvalitetssikringssystem og producenten skal have et post market surveillance system. Derudover skal der vælges et bemyndiget organ, som godkender, at dokumentationen lever op til gældende lovgivning. Disse krav er ikke blevet implementeret i Automatisk Ultralydsscanner, da der var været mest fokus på udviklingsprocessen af Automatisk Ultralydsscanner. 

%På Aarhus Universitetshospital, Tage-Hansens Gade, var personalet positive overfor en automatiseret ultralydsscanner, da det kan give økonomisk mening i forhold til at afhjælpe problematikken med manglende radiologer. De potentielle patienter som svarede på spørgeskemaundersøgelsen var generelt positive overfor en automatiseret ultralydsscanner, men de fremhævede vigtigheden ved robottens følsomhed overfor tryk. Der blev det formodet, at en automatiseret ultralydsscanning vil kunne give økonomisk mening med kortere ventetider, spare tid og dermed frigørelse af ressourcer i form af personale til andre opgaver. 