\chapter{Resumé}
\textbf{Baggrund} \newline
I Danmark tilbydes alle kvinder i alderen 50 til 69 år en rutinemæssig mammografiscreening. Mammografi med røntgen kan være uhensigtsmæssig at anvende, da det for kvinder med meget kirtelvæv, er svært at skelne kirtelvæv fra kræftknuder. Derfor er man i nogle tilfælde nødt til at supplere med en ultralydsscanning \cite{Ultralyd}.

Mammografi foretages i dag af radiografer eller røntgensygeplejersker, hvorefter røntgenbillederne sendes til en radiolog, som afgør om der skal foretages yderligere undersøgelser. I fremtiden kunne man forestille sig, at automatiserede ultralydsscanninger til screening for brystkræft kunne foretages med samme arbejdsprocedure, som man i dag udfører mammografi.

\textbf{Metoder} \newline
I udviklingsprocessen er Scrum brugt til at organisere projektet. Til undersøgelse af hvilke tiltag der skal til for at realisere Automatisk Ultralydsscanner er der lavet en spørgeskemaundersøgelse, en medicinsk godkendelse og en økonomiske analyse samt litteratursøgning. Til design af Automatisk Ultralydsscanner er UML og SysML anvendt. 

\textbf{Resultat} \newline
Systemet Automatisk Ultralydsscanner består af en PC applikation der vha. et 3D kamera instruerer en robotarm til at dække overfladen på et brystområde. Automatisk Ultralydsscanner kan ikke lave en fuld ultralydsscanning, og er ikke medicinsk godkendt. Tilføjelse af ultralydsscanninger til screeningsprogrammet er sandsynligvis ikke omkostningseffektivt.

\textbf{Diskussion} \newline
For at Automatisk Ultralydsscanners procedure-kvalitet skal på samme niveau som en radiologs, er det nødvendigt at optimere en række implementeringer. Derudover stilles der spørgsmålstegn ved indførslen af systemet i sundhedsvæsenet.

\textbf{Konklusion} \newline
Der blev udviklet et system, som til dels opfylder kravspecifikationen ift. afgrænsning af systemet. Der konkluderes, at en automatisk ultralydsscanner til screening for brystkræft er realistisk at udvikle.