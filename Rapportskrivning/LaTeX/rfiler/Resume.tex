\chapter{Resumé}
\textbf{Baggrund}
I Danmark tilbydes alle kvinder i alderen 50 til 69 år en rutinemæssig mammografiscreening. Mammografimetoden kan være uhensigtsmæssig at anvende, da det, for kvinder med meget kirtelvæv, er svært at se skelne kirtelvæv fra kræftknude, på et røntgenbillede. Derfor er man i nogle tilfælde nødt til at supplere med en ultralydsundersøgelser \cite{Ultralyd}.

I fremtiden kan man forestille sig, at automatiserede ultralydsscanninger til screening for brystkræft kunne foretages med samme procedure, som man i dag udfører mammografi. Mammografi foretages af radiografer eller røntgensygeplejersker, hvorefter røntgenbillederne sendes til en radiolog, som afgør, om der skal foretages yderligere undersøgelser.

Dette bachelorprojekt handler derfor om udvikling og undersøge af muligheden for at lave automatiserede ultralydsscanninger til screening for brystkræft, som supplement til mammografi.

\textbf{Metoder}
I udviklingsprocessen er der lavet en spørgeskemaundersøgelse, Medicinsk godkendelse og en økonomiske analyse for at undersøge hvilke tiltag der skal til for at kunne realisere Automatisk Ultralydsscanner. Til at beskrive systemet bag Automatisk Ultralydsscanner er UML og sysML anvendt. 

\textbf{Resultat}
Der er udviklet en PC Applikation, som gør det muligt for 3D Kamera at konstruere en 3D model, som Robotarm efterfølgende kan føre ultralydssproben over. 

\textbf{Konklusion}
Der blev udviklet et system, som automatisk kan udføre screeninger af brystet. Manglende kendskab til MDD har gjort at Automatisk Ultralydsscanner ikke overholder loven til medicinsk godkendelse. 