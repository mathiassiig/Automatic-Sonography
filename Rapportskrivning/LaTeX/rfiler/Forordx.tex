\chapter{Forord}
Dette bachelorprojekt er udarbejdet af Marie Kirkegaard, Charlotte Søgaard Kristensen og Mathias Siig Nørregaard og er det afsluttende projekt på 7. semester for diplomingeniøruddannelsen i henholdsvis Sundhedsteknologi og informations- og kommunikationsteknologi ved Aarhus School of Engineering (ASE). Bachelorprojektet er udarbejdet i perioden 29. august 2016 til 16. december 2016 og ideen bag bachelorprojektet er opstået hos Søren Holm Pallesen, stifter og direktør for Robotic Ultrasound. 

Formålet med projektet er at undersøge muligheden for at lave automatiske og reproducerbare ultralydscanninger til screening for brystkræft. Bachelorprojektet henvender sig primært til ingeniører indenfor sundhedsteknologier og informations- og  kommunikationsteknologi, samt undervisere og studerende på ovenstående uddannelser. 

Bachelorgruppen vil gerne rette en stor tak til alle som har hjulpet med sparring gennem hele projektets forløb:

\begin{itemize}
\item Cand.Polyt. og lektor Michael Alrøe for support og engageret vejledning gennem projektforløbet.

\item Cand.scient.med. Søren Holm Pallesen for at stå til rådighed ved diverse relevante
møder, og for altid at have gode ideer til projektarbejdet.

\item Overlæge og lektor Lars Bolvig Hansen for telefoninterview og teknisk sparring på ultralyds området.

\item Afdelingsradiograf Tine Bisgaard for besøg og rundvisning på Røntgen- og Skanningsafdelingen på Aarhus Universitetshospital, Tage Hansen Gade. 
\end{itemize}

\section{Læsevejledning} 
Bachelorprojektet er delt op i en projektrapport og procesrapport. Projektrapporten beskriver selve projektet, hvordan problemformuleringen er det løst, samt hvilke resultater er der opnået. Procesrapporten, som findes umiddelbart efter projektrapporten, beskriver hele udviklingsprocessen af Automatisk Ultralydsscanner. Dette inkluderer projektadministrationen, arbejdsfordeling, planlægning og brug af projektstyringsværktøjer. 

I hvert afsnit er der refereret til bilag, hvor man kan finde uddybende dokumentation. Bilagslisten kan findes nederst i hver rapport.   
