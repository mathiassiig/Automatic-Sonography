\chapter{Abstract}\label{kapitel_Abstract}
\textbf{Background} \newline
All Danish women between the age of 50 to 69 years are offered 
mammography to screen for breast cancer. This method of screening may sometimes be unsuitable, as it is hard to distinguish between glandular tissue and tumors in an x-ray image - and some women have a lot of glandular tissue. In these cases it is necessary to additionally perform an ultrasonography procedure.

Nowadays, mammography procedures are carried out by radiographers or radiology nurses, after which the x-ray images are sent to a radiologist. The radiologist then decides if additional examinations are required. It is possible to envision automatic ultrasonographic procedures which could be accomplished with the same work flow as mammography procedures.

\textbf{Methods} \newline
Elements of Scrum was used to organize the project in the development process. A user survey, a medical approval draft and an economic analysis have been produced to investigate which approaches are needed to realize Automatic Sonography. UML and SysML have been used to describe Automatic Sonography.

\textbf{Results} \newline
The system Automatic Sonography includes a PC application which utilizes a 3D camera to instruct a robot arm to move across the surface of a chest area. Automatic Sonography cannot perform a full sonographic procedure and is not medically approved. A sonographic addition to the screening program is not necessarily cost-effective.

\textbf{Discussion}\newline
It is necessary to introduce a range of optimizations for Automatic Sonography to mimic the quality of a radiologist's scanning procedure. Additionally, the introduction of the system to the Danish healthcare service remains questionable.

\textbf{Conclusion} \newline
A system, which partially meets the requirements specified according to the delimitation, has been developed. In conclusion, it is realistic to develop an automatic sonography system to screen for breast cancer.