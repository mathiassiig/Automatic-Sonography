\chapter{Test}\label{Test}
I dette afsnit beskrives, hvordan systemet er testet for at sikre, at design og implementeringen lever op til systemkrav, defineret i kravspecifikation bilag \ref{Kravspecifikation}. Der er både udført en accepttest (bilag \ref{Accepttest}) og unittests af softwaren i PC Applikation. Læs nærmere om unittests i afsnittet 'Test' i bilag \ref{Udviklingsdokument} Dokumentation.

\section{Accepttest}

Accepttesten er designet for at teste funktionelle og ikke-funktionelle krav, som er beskrevet i bilag \ref{Kravspecifikation} Kravspecifikation. Testen udføres typisk sammen en kunde, men i bachelorprojektet er accepttesten udført med vejleder, Michael Alrøe.  Fejl og årsager er nærmere beskrevet i bilag \ref{BilagAccepttest} Accepttest.  

Det er forsøgt at designe hver test til, at hvert trin i kravspecifikationen testes og har et visuelt og reproducerbart resultat. Dette gør sig gældende for UC1: Start system, UC2: 3D scan brystområde, UC4: Stop system og de ikke-funktionelle krav, men hovedscenariet i UC3: Ultralydsscan brystområde har været problematisk at designe. Succeskriterierne for UC3 var at få en ultralydsscanning af Patients brystområde, men da der ikke har været en ultralydsscanner eller radiolog tilrådighed, har det været svært at validere Automatisk Ultralydsscanners evne til at ultralydsscanne. Primært af reproducerbare testårsager, samt sikkerhedsårsager, er der til accepttesten benyttet et 3D-printet testobjekt, udformet som et bryst, som erstatning for patient. Det har været forsøgt at markere banen på testobjektet ved at benytte forskellige metoder: maling på testobjekt, tape-beklædning og tusch-anvendelse samt ler og en affjedret pind. Ingen af løsningerne virkede. Det endte med, at der ikke blev anvendt en probe til markeringen af banen, men at testpersonen blot måtte observere og vurdere banen over testobjektet. 
\newpage
\subsection{Funktionelle krav} 
Test af funktionelle krav inkluderer syv forskellige test, da både hovedforløb, udtagelser, og udvidelser i de fire use cases skal testes. Nedenstående tabel viser de forskellige tests, der er udført. 

\begin{table}[htb]
\centering
\begin{tabular}{ | l | p{0.80\textwidth} | } 
\hline
\textbf{Testens navn} \\\hline
UC1: Hovedscenarie\\\hline 
UC2: Hovedscenarie \\\hline 
UC2: Undtagelse: Juster 3D billedets skæring \\\hline 
UC3: Hovedscenarie \\\hline 
UC3: Udvidelse: Operatør pauser scanning \\\hline 
UC3: Undtagelse: Operatør stopper scanning \\\hline 
UC4: Hovedscenarie \\\hline 
\end{tabular}
\caption{Test af funktionelle krav} 
\end{table}

Se bilag \ref{Accepttest} Accepttest for det fulde testsetup og resultater af hver test. 

\subsection{Ikke-funktionelle krav} 
De ikke-funktionelle krav er MoSCoW-metoden benyttet til at prioritere kravenes vigtighed. Det er valgt kun at teste ’must’-kravene, som kan ses i nedenstående tabel. 

\begin{table}[htb]
\centering
\begin{tabular}{ | l | p{0.80\textwidth} | }
\hline
\textbf{Type} & \textbf{Navn} \\\hline
Usability & U1. PC Applikation skal have en GUI \\\hline 
Performance & P1. Scanningen med 3D kamera og ultralydsscanning skal max tage 10
minutter til sammen \\\hline 
Performance & P2. Startoptid på PC Applikation skal være max 30 sekunder \\\hline
Performance & P3. 3D kamera skal max bruge 1 minut om at tage 3D billedet \\\hline 
Performance & P4. PC Applikation skal max bruge 1 minut på at færdiggøre brystområdets
positurer til Robotarm \\\hline 
\end{tabular}
\caption{Test af ikke-funktionelle krav}
\end{table}

Se bilag Accepttest for det fulde testsetup og resultater for hver test. 

\subsection{Automatiserede tests af PC Applikation}
Der er undervejs i udviklingen af PC Applikation skrevet automatiserede tests af softwarens forskellige moduler. Testene er primært unittests, hvor hvert modul testes som enhed. Nogle moduler, specifikt dem der direkte afhængige af hardware, er ikke blevet testet vha. unittests. Se bilag \ref{Udviklingsdokument} Dokumentation, afsnit 'Test' for en uddybning.