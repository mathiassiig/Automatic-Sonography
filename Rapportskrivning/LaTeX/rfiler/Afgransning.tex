\chapter{Afgrænsning}
\label{Afgransning}

Projektet har ikke haft en ultralydsscanner til rådighed under udviklingen. Dette har heller ikke har været essentielt for at få et succesfuld projektforløb, da  Automatisk Ultralydsscanner blot har ultralydsscannerens probe monteret mekanisk. Samtidig blev det i udviklingsforløbet besluttet ikke at inkludere eksterne sensorer eller software til målingen af påført tryk. Det er i stedet valgt at prioritere 3D kameraets genkendelse af dybde i brystområdet, og derefter bruge billedet til at få robotarmen til at bevæge sig rundt på det detekterede område. Det blev bedømt at disse elementer var vigtigere at implementere, for at få et essentielt minimum viable product.

Der er i projektet blevet benyttet et 3D kamera af typen Kinect 2.0, som er et forholdsvis et stort kamera. Det blev vurderet at det ville være besværligt at have både en ultralydsprobe og en Kinect monteret på robotarmen på samme tid. Det er derfor valgt at montere 3D kameraet i loftet. 

Ved en konventionel ultralydsscanning scannes armhulerne også. Grundet besværligheden i visuelt at skulle afgrænse 3D scanningen til også at få armhuler med uden at scanne uinteressante områder, blev 3D scanningen afgrænset til kun at inkludere brystområdet.