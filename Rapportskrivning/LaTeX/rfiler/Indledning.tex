\chapter{Indledning}
\section{Baggrund}
Danmark vil i de kommende årtier få en voksende andel af ældre borgere, der vil lægge et større pres på velfærdssamfundet i Danmark. Det vil resultere i færre borgere i den arbejdsdygtige alder, da det ud fra en befolkningsfremskrivning til 2040 forventes, at andelen af personer på 65 år og derover vil udgøre omkring en fjerdedel af den samlede danske befolkning \cite{Befolk}. En sådan ændring i demografien vil betyde flere patienter med kroniske lidelser, som derved vil medføre et øget pres på og øgede omkostningerne for sundhedsvæsen \cite{Pres}. Det er derfor nødvendigt, at se på optimerede løsninger til behandling og diagnosticering af sygdomme. 

I Danmark tilbydes alle kvinder i alderen 50 til 69 år en rutinemæssig mammografiscreening. Mammografiscreening foregår ved en røntgenundersøgelse, hvilket er en billig og effektiv metode \cite{Afsloring}. Metoden er dog ikke altid den mest hensigtsmæssige at anvende, da kirtelvæv og ondartede cancersvulster kan være svære at skelne fra hinanden på et røntgenbillede. Røntgenmetoden har derfor en begrænset effekt på kvinder med meget kirtelvæv. Til disse patienter suppleres røntgenbillederne med en ultralydsundersøgelse, som fortages af en radiolog \cite{Ultralyd}.

Ultralyds- og røntgenundersøgelser har hver sine fordele og kan derfor sjældent stå alene. Ultralyd har den største diagnosesikkerhed i kirtelvæv, hvor røntgen har den største diagnosesikkerhed i fedtvæv. Da brystet ofte er en kombination af de to vævstyper, supplerer disse to metoder hinanden godt \cite{Ultralyd}. 

Mammografi foretages i dag af enten en radiograf eller en røntgensygeplejerske, hvorefter røntgenbillederne bliver sendt videre til en radiolog. I fremtiden kunne det forestilles, at automatiserede ultralydsscanninger til screening for brystkræft kunne foretages med samme procedure, hvor ultralydsvideoclips efterfølgende sendes til lægen.

Dette bachelorprojekt går derfor ud på at lave et Proof of Concept, for af undersøge muligheden for automatiserede ultralydsscanninger til screening for brystkræft.
\newpage

\section{Problemformulering}
Med udgangspunkt i projektets baggrund og projektbeskrivelse. Se bilag  \ref{Projektbeskrivelse} Projektbeskrivelse (ASE), er der defineret følgende problemstillinger, som forsøges besvaret og belyst i dette bachelorprojekt:

\textbf{Hvordan kan en automatiseret ultralydsscanner til screening for brystkræft udvikles samt hvilke økonomiske og produktsikkerhedsmæssige tiltag vil kunne realisere dette?}

\let\labelitemi\labelitemii
\begin{itemize}
\item Hvordan vil en automatiseret ultralydsscanning til screening for brystkræft kunne udvikles ved brug af robotarm og 3D kamera?
\item Hvilke omkostninger vil indførslen af en Automatisk Ultralydsscanner kunne give? 
\item Hvilke konsekvenser vil en tilføjelse af Automatisk Ultralydsscanner til screeningsprogrammet have?
\item Hvad kræves for at få Automatisk Ultralydsscanner CE-mærket? 
\end{itemize}
