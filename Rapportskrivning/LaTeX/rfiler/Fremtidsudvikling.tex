\chapter{Fremtidig udvikling}\label{kapitel_Fremtidig udvikling}
For at Automatisk Ultralydsscanner, skal kunne udføre automatiske ultralydsscanninger, er der en række ting man kunne implementere. Foruden de problemstillinger nævnt i diskussionsafsnittet kan følgende punkter implementeres:
Software (og evt. hardware) til tryk-korregering. Dette vil give mulighed for at variere det tryk, som Robotarm leverer på brystet. Trykvariation er nødvendig for at udføre en ultralydsscanning, da ultralydsproben kun kan lave en brugbar ultralydsscanning ved et bestemt tryk. I nuværende version vil Robotarm stoppe, hvis den trykker for hårdt - altså hvis trykket kommer over den tærskelværdi, der er givet. Samtidig vil Robotarm heller ikke tage højde for, hvis den slet ikke leverer noget tryk. I tilfælde af at tryk-værdierne fra Robotarm ikke er præcise nok, kan man være nødsaget til at påmontere en ekstern tryksensor på Robotarm.

En kollisions-forudsigelses-algoritme. For enhver position og rotation af TCP findes der i nogle tilfælde forskellige løsninger hvorpå Robotarm kan opnå den positur. Derfor vil der ved nogle af disse positurer være tilfælde hvor robot-led enten kolliderer med Patient eller Robotarm selv. Derfor bør rotationen af hvert led udregnes ud fra TCP positur, således at de ledrotationer hvor der ikke er kollision anvendes. Dette vil ændre måden der kommunikeres med Robotarm, og derfor vil det være nødvendigt at omskrive software ved den grænseflade, hvor Robotarm instrueres i at bevæge sig.

Et tredje problem er måden hvorpå positurer sendes til Robotarm. Imellem hver positur foregår en acceleration og deceleration. Disse accelerationer og decelerationer giver uønskede rykvise bevægelser. Ved en ultralydsscanning, foretaget af en radiolog, vil der kun være acceleration og deceleration for hvert bryst der scannes, og ikke imellem N antal punkter i brystet der scannes.

For at Automatisk Ultralydscanner skal kunne foretage en fuld ultralydsscanning, vil det også være nødvendigt med flere vinkler fra 3D kamera. Automatisk Ultralydsscanner er begrænset til at kunne scanne det område af brystet, som 3D kamera kan detektere. I nuværende version er 3D kamera monteret fast til loftet, og derfor er der blinde vinkler på siden, af brystet der skal scannes. Her vil det være ideelt, at Robotarm har 3D kamera påmonteret, og flere 3D scanninger fra samme 3D kamera kan kombineres. Den nuværende prototype ignorerer pejlemærkerne i starten og slutningen af scanningen, da der er utilstrækkelig information om 3D scanningen fra siderne.

I fremtiden kunne det også være en stor fordel, hvis man kunne anvende Automatisk Ultralydsscanner til andre simple procedurer. For eksempel flowmåling i arme og ben, hvilket formentligt ikke vil kræve andet hardware end det, der er anvendt i bachelorprojektet.