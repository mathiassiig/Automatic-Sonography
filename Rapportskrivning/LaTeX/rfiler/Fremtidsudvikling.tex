\chapter{Fremtidig udvikling}\label{kapitel_Fremtidig udvikling}

Næste skridt i udviklingen af Automatisk Ultralydsscaner vil være software til tryk-korrigering.
Dette vil give mulighed for at variere det tryk, som Robotarm leverer på brystet. Trykvariation er nødvendig for at udføre en ultralydsscanning, da ultralydsproben kun kan lave en brugbar ultralydsscanning ved et bestemt tryk. 
I nuværende version vil Robotarm stoppe, hvis den trykker for hårdt, altså hvis trykket kommer over den tærskelværdi, der er givet. Samtidig vil Robotarm heller ikke tage højde for, hvis den slet ikke leverer noget tryk.
Det kan være, at tryk-værdierne fra Robotarm ikke er brugbare nok, og man derfor ud over software ville være tvunget til at bruge en tryksensor.

For at Automatisk Ultralydscanner skal kunne foretage en fuld scanning, vil det også være nødvendigt med flere vinkler fra 3D kamera. Automatisk Ultralydsscanner er begrænset til at kunne scanne det område af brystet, som 3D kameraet kan detektere. I nuværende version er 3D kamera monteret fast til loftet. Her vil det være ideelt, at Robotarm har 3D kamera monteret, så flere 3D billeder fra samme 3D kamera kan kombineres.

Automatisk Ultralydsscanner blev udviklet med det formål at udføre en ultralydsscanning til en simpel procedure, hvor radiologen i mindre grad var en del af systemet. Derfor kunne der være et scenario i fremtiden, hvor systemet selv kan finde knuder gennem billedgenkendelse. Derved vil radiologen kunne varetage vigtigere opgaver, og patienten kan få svar allerede inde ved konsultationen i stedet for at vente.

Hvis Automatisk Ultralydsscanner skal anvendes på hospitaler, vil det være nødvendigt at integrere et login til operatør, samt lave identifikation i form af CPR-nummer i PC Applikationen. Systemet skal derfor også kunne gemme målinger sikkert i en database. 

I fremtiden kunne det også være en stor fordel, hvis man kunne anvende Automatisk Ultralydsscanner til andre simple procedurer. For eksempel flowmåling i arme og ben, hvilket formentligt ikke vil kræve andet udstyr end det, der er anvendt i bachelorprojektet. En Automatisk Ultralydsscanner kan i sin nuværende form ikke anvendes til scanninger in vivo.

Det vil også være nødvendigt at udføre kravene til medicinsk godkendelse for at få Automatisk Ultralydsscanner CE-mærket. Dette vil blandt andet indebære udvikling af kvalitetssikringsystem, risikovurdering, overholdelse af væsentlige krav fra MDD. Derudover skal der udføres test for Elektromagnetisk stråling, bio-kompatibilitetstestes og udføre en klinisk test på Automatisk Ultralydsscanner, samt lave en klinisk evaluering. 
