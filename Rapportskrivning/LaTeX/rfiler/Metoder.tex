\chapter{Metoder}\label{Metoder}

\section{Udviklingsværktøjer} 
Til udviklingen af dokumentation for Automatisk Ultralydsscanners design, er der brugt arbejdsværktøjerne Unified Modeling Language (UML)  og Systems Modeling Language (SysML). Projektarbejdet har bestået af udvikling af software, hvor  UML er benyttet til at opstille bl.a. use cases, aktør-kontekst diagrammer samt sekvens- og klassediagrammer mm. Systemet anvender allerede udviklet hardware, som PC applikationen skal have forbindelse til. SysML er benyttet til at lave block definition diagram (BDD) og internal block diagram (IBD) til at illustrere forbindelsen mellem de forskellige komponenter som f.eks. robotarmern.

\section{Kvalitative og kvantitative undersøgelser}
Der er brugt både kvalitative og kvantitative undersøgelser til belysning af projektets problemstillinger. Den kvalitative metode er anvendt i til interviews af overlæge og radiolog Lars Boldvig og afdelingsradiograf Tine Bisgaard for at undersøge praksis ved ultralydsscanninger. Den kvantitative metode,  er benyttet i sammenhæng med den kvalitative i forbindelse med spørgeskemaundersøgelsen af potentielle patienter for at undersøge, hvordan hvordan og hvad der skal til før patienter vil tage imod en automatiseret robotarm til ultralydsscanninger.  

\section{Økonomisk og omkostningseffektiv analyse}
Der er udarbejdet en økonomisk analyse, da det er væsentligt grundlag for indførsel af ny teknologi i sundhedssektoren. Analysen indeholder omkostningerne forbundet med implementering af den Automatiske Ultralydsscanner, og hvilken ækonomiske konsekvenser, en udvidelse af screeningsprogrammet, vil have. 

\section{Litteratursøgning} 
Det er undersøgt, hvilke andre konsekvenser en udvidelse af screeningsprogrammet til screening for brystkræft vil medføre. Der er søgt med emneord inden for problemformuleringen i både national og international litteratur på de større søgebaser for at undersøge, om tidlig detektering af brystkræft er rentabel og omkostningseffektiv. 

\section{Medicinsk godkendelse}
Til udarbejdelsen af medicinsk godkendelse af Automatisk Ultralydsscanner er der benyttet direktiver og tilhørende standarder til at beskrive vejen til at få Automatisk Ultralydsscanner CE-mærket. Selve godkendelsesproceduren er ikke gennemført, men blot beskrevet for at give et overblik over hvordan godkendelsesproceduren kan gennemføres. 

