\chapter{Metoder}\label{Metoder}

Til udviklingen af dokumentation for Automatisk Ultralydsscanners design, er der brugt arbejdsværktøjerne Unified Modeling Language (UML) \cite{UML}  og Systems Modeling Language (SysML) \cite{SysML}. Projektarbejdet har bestået af udvikling af software, hvor UML er benyttet til at opstille bl.a. use case-, aktør-kontekst-, sekvens-, pakke-, tilstands- og klassediagrammer. Systemet anvender allerede udviklet hardware, som PC Applikation skal have forbindelse til. SysML er benyttet til at lave block definition diagram (BDD) og internal block diagram (IBD) til at illustrere forbindelsen mellem de forskellige komponenter som f.eks. Robotarm.

Der er anvendt kvalitative og kvantitative metoder \cite{MetoderBruger} til brugerundersøgelser, hvor den kvalitative primært er brugt til at øge forståelsen omkring projektets emne, mens den kvantitative er brugt i et spørgeskema til potentielle patienter. En break-even analyse \cite{Erhvervsokonomi} er benyttet som værktøj til at give en forståelse for, hvordan omkostninger ændres ift. forandringer af transportvariabler. Litteratursøgning med søgeprotokol er benyttet til at finde litteratur om screeningsprogrammer. Se afsnit \ref{Analyser} Analyser, for anvendelsen af metoderne. 

Til udarbejdelsen af medicinsk godkendelse af Automatisk Ultralydsscanner, er Medical Device Directive 93/42/EØF (MDD)\cite{MDD} benyttet til at beskrive, hvordan godkendelsesproceduren for CE-mærkning kan gennemføres. Se afsnit \ref{MedicinskGodkendelse} Medicinsk godkendelse.

Til udarbejdelse af produktet og tilhørende krav, design og test, er der blevet anvendt elementer af Scrum og V-modellen. For at læse mere om brugen af disse værktøjer, se procesrapporten, der følger denne rapport.

