\chapter{Metoder}\label{Metoder}

\section{Udviklingsværktøjer} 
Til udviklingen af dokumentation for Automatisk Ultralydsscanners design, er der brugt arbejdsværktøjerne Unified Modeling Language (UML) \cite{UML} og Systems Modeling Language (SysML) \cite{SysML}. UML er en standard for diagrammer og beskriver objektorienteret programmering, mens SysML er en udvidelse af UML og er et grafisk værktøj til at modellere systemer, hvori der også kan indgå hardware. Projektarbejdet har bestået af udvikling af software, hvor  UML er benyttet til at opstille bl.a. use cases, aktør-kontekst diagrammer samt sekvens- og klassediagrammer mm. Systemet anvender allerede udviklet hardware, som PC applikationen skal have forbindelse til. SysML er benyttet til at lave block definition diagram (BDD) og internal block diagram (IBD) til at illustrere forbindelsen mellem de forskellige komponenter som f.eks. robotarmern.

\section{Kvalitative og kvantitative undersøgelser}
Der er brugt både kvalitative og kvantitative undersøgelser til belysning af projektets problemstillinger. Den kvantitative metode som undersøger i bredden er benyttet i forbindelse med spørgeskemaundersøgelsen af potentielle patienter. Den kvalitative metode som undersøger i dybden og er anvendt i form af interviews, hvor både overlæge og radiolog Lars Boldvig og afdelingsradiograf Tine Bisgaard er blevet interviewet. 

\section{Økonomisk og omkostningseffektiv analyse}
Der er udarbejdet en økonomisk og omkostningseffektiv analyse, da det er væsentligt grundlag for indførsel af ny teknologi i sundhedssektoren. Analysen indeholder omkostningerne forbundet med implementering af den Automatiske Ultralydsscanner, og hvilke andre konsekvenser, en udvidelse af screeningsprogrammet, vil have. 

\section{Medicinsk godkendelse}
Til udarbejdelsen af medicinsk godkendelse af Automatisk Ultralydsscanner er der benyttet direktiver og tilhørende standarder til at beskrive vejen til at få Automatisk Ultralydsscanner CE-mærket. Selve godkendelsesproceduren er ikke gennemført, men blot beskrevet for at give et overblik over hvordan godkendelsesproceduren kan gennemføres. 