\chapter{Diskussion}\label{kapitel_Diskussion}
Med Automatisk Ultralydsscanner er det opnået at lave et system, der kan lave et 3D billede af et testobjekt og derefter få Robotarm til at rotere omkring dette. Alle funktionelle og ikke-funktionelle krav på nær hovedscenariet i UC3: Ultralydsscan brystområdet blev godkendt i accepttesten. 

Det var ikke muligt for Robotarm at udføre bevægelser i det specifikke bevægelsesmønster, beskrevet i Bilag \ref{Kravspecifikation} Kravspecifikation, hvilket gjorde, at UC3: Hovedscenarie fejlede accepttesten. Det betyder ikke, at Robotarm ikke kan bevæge sig over det område, som 3D kamera har detekteret, men at det ikke har været muligt at implementere det specifikke bevægelsesmønster nævnt på side \pageref{Probensbevagelse}. Robotarms bane over brystet er begrænset af 3D kameras detektering af overfladen. I detekteringen kan der opstå ujævnheder, og derfor er der i PC applikation forsøgt at lave en gennemsnitlig overflade med en udglatningsfunktion for at undgå ujævnheder, da 3D kamera er upræcist. Dette resulterer i, at de punkter Robotarm benytter som pejlemærker også er upræcise. For at Robotarms bevægelser kan blive præcise, vil det være nødvendigt at anvende et mere præcist 3D kamera, bedre udjævningsalgoritmer samt en mere nøje bestemmelse af stien Robotarm skal dække. Derudover vil det være en stor fordel at montere kameraet på Robotarms yderste led, da dette vil give mulighed for at se testobjektet fra blinde vinkler. 

For at bestemme rotationen af Robotarms Tool Center Point (TCP), er der lavet pitch, yaw og roll beregninger, som er implementeret i PC Applikation. En bivirkning af beregningen er, at Robotarms yderste led roterer. Dette er uønsket af to grunde: En radiolog vil aldrig rotere en ultralydsprobe på denne måde, og det betyder, at Robotarm kan over- eller underrotere dette led. Dette resulterer i, at Robotarm stopper op i løbet af ultralydsscanningen, men kan startes igen, ved manuelt at rotere det yderste led tilbage. Det er endnu ikke lykkedes at finde en løsning på dette problem.

Det var nødvendigt at give Operatør mulighed for at kunne se og godkende billedet fra 3D kamera, da der er tilfælde, hvor 3D scanningen ikke bliver udført som forventet. Derfor blev det valgt at vise dybdebilledet og 3D scanningen i selve GUI'en. Med disse to billeder vil Operatør nemmere kunne justere på værdier, der afgrænser området, der skal scannes, for ikke at ultralydsscanne noget, der ikke er brystvæv.

I projektet er performancetiderne til systemet blevet prioriteret højt, da det potentielt vil kunne forbedre mulighederne for implementering af Automatisk Ultralydsscanner i sundhedsvæsenet, hvor tid er en vigtig resurse. Til accepttesten tog det Automatisk Ultralydsscanner omkring 3 minutter at afvikle først 3D scanning og derefter rotere Robotarm rundt på det detekterede område, hvor det i kravspecifikationen bilag \ref{Kravspecifikation} Kravspecifikation, er et krav, at det må tage 10 minutter til sammen. En 3D scanning er sat til at må tage 10 sekunder, men til accepttesten tog det omkring 2 sekunder at vise 3D billedet på GUI. PC Applikation brugte under 1 sekund, hvor kravet var 10 sekunder, på at færdiggøre brystområdets positurer til Robotarm. Disse krav er alle defineret ud fra interviews med radiolog Lars Bolvig og afdelingsradiograf Tine Bisgaard, hvor de gav deres bud på, hvor lang tid en ultralydsscanning tager. Validiteten kan her diskuteres, da det kun er udtalelser fra to personer.
Performance-tiderne lå generelt et godt stykke under de definerede krav, defineret ud fra interviews med Lars Bolvig og  Tine Bisgaard, men der er flere faktorer der spiller ind. Performance-tiden er meget afhængige af de beslutninger, der tages for hvor høj kvaliteten af ultralydsscanningen skal være. En højere opløsning på 3D scanningen samt flere gennemgange af udjævnings-algoritmen vil øge den tid PC Applikation bruger på at finde positurer, betydeligt.  

Der er mangel på radiologer i Danmark, såvel som resten af verden \cite{Lagemangel}, hvorfor man skal tænke andre metoder og arbejdsgange. På længere sigt kan man forestille sig, at radiografer overtager nogle af opgaverne fra radiologerne, heriblandt bl.a. ultralydsscanninger. Ved f.eks. at lade radiografer lave selve scanningerne, vil patienter først få besked efter radiologen har tilset scanningerne. Den procedure er også brugt til mammografi med røntgen. Det kan diskuteres, om en udvidelse af screeningsprogrammet vil være en god ting: Udvidelsen vil give mulighed for at finde flere kræfttilfælde - men der er risiko for overbehandling og øgede udgifter til sundhedsvæsenet. Man kunne i stedet forestille sig, at en del af røntgenscanningerne kunne erstattes af ultralydsscanninger, da 1 ud af 100.000 \cite{RiskRontgen} patienter udvikler kræft af røntgenstråling, mens der ikke er nogen kendte bivirkninger ved ultralydsscanninger.

Den medicinske godkendelse burde have været implementeret i kravspecifikationens funktionelle krav til Automatisk Ultralydsscanner, for at overholde lovgivning til CE-mærkning. Det blev besluttet at prioritere udviklingen af Automatisk Ultralydsscanner højere end den medicinske godkendelse, da bachelorprojektet skulle undersøge selve muligheden for at udvikle et system til automatiseret ultralydsscanning af mamma mhp. screening for brystkræft. Den medicinske godkendelse vil derfor kunne have været en forhindring i denne udviklingsproces, og andre krav ville ikke være blevet implementeret. 
Nogle af de krav der skal til for at få Automatisk Ultralydsscanner CE-mærket, vil blandt andet indebære udvikling af kvalitetssikringsystem, risikohåndtering, overholdelse af væsentlige krav fra MDD. Derudover skal der udføres test for Elektromagnetisk stråling, bio-kompatibilitetstestes og udføre en klinisk test på Automatisk Ultralydsscanner, samt lave en klinisk evaluering. Det man ville starte med, vil være at undersøge om hvert komponent, er godkendt til medicinsk brug, før udviklingen påbegyndes. Det kan tage mange år at få udstyr godkendt til medicinsk brug, derfor er det en stor fordel at anvende allerede godkendte komponenter. Robotarm UR10 er f.eks ikke godkendt til medicinsk brug\ref{UserManualUR10}, derfor vil det være nødvendigt at udskifte UR10 med en godkendt Robotarm. 
