\chapter{Resultater}\label{kapitel_Resultater}

\section{Automatisk Ultralydsscanner}
I accepttesten blev de funktionelle og ikke-funktionelle krav for Automatisk Ultralydsscanner testet. Resultatet af accepttesten var at alle test blev godkendte, på nær UC3: Hovedscenarie. Se Tabel \ref{funk} og \ref{ikke}. 

\subsection{Funktionelle krav}
\begin{table}[htb]
\centering
\begin{tabular}{ | l | p{0.20\textwidth} | }
\\\hline
\textbf{Testens navn} & \textbf{Godkendt} \\\hline
UC1: Hovedscenarie & \checkmark \\\hline 
UC2: Hovedscenarie & \checkmark \\\hline 
UC2: Undtagelse: Juster 3D billedets skæring & \checkmark \\\hline 
UC3: Hovedscenarie & - \\\hline 
UC3: Udvidelse: Operatør pauser scanning & \checkmark \\\hline 
UC3: Undtagelse: Operatør stopper scanning & \checkmark \\\hline 
UC4: Hovedscenarie & \checkmark \\\hline 
\end{tabular}
\label{funk}
\caption{Funktionelle krav} 
\end{table}

\subsection{Ikke-funktionelle krav}
\begin{table}[htb]
\centering
\begin{tabular}{| l | p{0.65\textwidth}| l |}
\hline
\textbf{Type} & \textbf{Navn} & \textbf{Godkendt}\\\hline
Usability & U1. PC Applikation skal have en GUI & \checkmark \\\hline 
Performance & P1. Scanningen med 3D kamera og ultralydsscanning skal max tage 10
minutter til sammen & \checkmark \\\hline 
Performance & P2. Startoptid på PC Applikation skal være max 30 sekunder & \checkmark \\\hline
Performance & P3. 3D kamera skal max bruge 1 minut om at tage 3D billedet & \checkmark \\\hline 
Performance & P4. PC Applikation skal max bruge 1 minut på at færdiggøre brystområdets
positurer til Robotarm & \checkmark \\\hline 
\end{tabular}
\label{ikke}
\caption{Ikke-funktionelle krav}
\end{table}

Det UC3: Hovedsenarie fejlede på, var punkt 2.1 hvor Testperson observerer, om Robotarm roterer omkring Testobjekt. Grunden til, at accepttesten fejlede var at det ikke var muligt at udføre en ultralydsscanning.  

Det vil sige at Automatisk Ultralydsscanner blev udviklet så det er muligt for Operatør at få forbindelse til PC Applikation, udføre en 3D scanning og dannelse af dybdebillede af brystet.  3D billedet kan justeres og ultralydsscanningen kan stoppes og pauses. 

For at detaljerede resultater af accepttest se bilag \ref{Accepttest} Accepttest. 