\chapter{Resultater}\label{kapitel_Resultater}
I dette afsnit er projektets resultat fra accepttesten beskrevet. De opnåede resultater er dokumenteret i accepttest-dokumentet og der henvises derfor til Bilag \ref{Accepttest} Accepttest, for detaljeret beskrivelse af accepttesten. 

\section{Automatisk Ultralydsscanner}
I accepttesten blev de funktionelle og ikke-funktionelle krav for Automatisk Ultralydsscanner testet. Resultatet af accepttesten var at alle tests blev godkendt, på nær UC3: Hovedscenarie. Se Tabel \ref{funk} og \ref{ikke}. 

\subsection{Funktionelle krav}
De funktionelle krav for Automatisk Ultralydsscanner blev godkendt ved test af use cases. 
\begin{table}[htb]
\centering
\begin{tabular}{ | l | p{0.20\textwidth} | }
\hline
\textbf{Testens navn} & \textbf{Godkendt} \\\hline
UC1: Hovedscenarie & \checkmark \\\hline 
UC2: Hovedscenarie & \checkmark \\\hline 
UC2: Undtagelse: Juster 3D billedets skæring & \checkmark \\\hline 
UC3: Hovedscenarie & - \\\hline 
UC3: Udvidelse: Operatør pauser scanning & \checkmark \\\hline 
UC3: Undtagelse: Operatør stopper scanning & \checkmark \\\hline 
UC4: Hovedscenarie & \checkmark \\\hline 
\end{tabular}
\caption{Funktionelle krav}\label{funk} 
\end{table}

Automatisk Ultralydsscanner blev udviklet så det er muligt for Operatør at interagere med PC Applikation, udføre en 3D scanning, og instruere Robotarm om at udføre en ultralydsscanning. 3D billedet kan afgrænses og ultralydsscanningen kan pauses og stoppes. 

Det var ikke muligt for Robotarm at køre i det specifikke bevægelsesmønster beskrevet i bilag \ref{Kravspecifikation} Kravspecifikation og dette var grunden til at UC3: Hovedsenarie fejlede i punkt 2.1 - Testperson observerer, om Robotarm roterer omkring Testobjekt i et specifikt bevægelsesmønster. 
\newpage

\subsection{Ikke-funktionelle krav}
De ikke-funktionelle krav for Automatisk Ultralydsscanner blev godkendt ved test af must-krav. 
 
\begin{table}[htb]
\centering
\begin{tabular}{| l | p{0.65\textwidth}| l |}
\hline
\textbf{Type} & \textbf{Navn} & \textbf{Godkendt}\\\hline
Usability & U1. PC Applikation skal have en GUI & \checkmark \\\hline 
Performance & P1. Scanningen med 3D kamera og ultralydsscanning skal max tage 10
minutter til sammen & \checkmark \\\hline 
Performance & P2. Startoptid på PC Applikation skal være max 30 sekunder & \checkmark \\\hline
Performance & P3. 3D kamera skal max bruge 1 minut om at tage 3D billedet & \checkmark \\\hline 
Performance & P4. PC Applikation skal max bruge 1 minut på at færdiggøre brystområdets
positurer til Robotarm & \checkmark \\\hline 
\end{tabular}
\caption{Ikke-funktionelle krav}\label{ikke}
\end{table}

Det vil sige at PC Applikation blev udviklet med en GUI, det var muligt at udføre en 3D scanning og ultralydsscanning på 3 minutter. Starttiden på PC Applikation var under 30 sekunder, det tog under et minut at tage et 3D billede og PC Applikation brugte under 1 minut på at færdiggøre positurer til ultralydsscanning af brystområdet.  

For at detaljerede resultater af accepttest se bilag \ref{Accepttest} Accepttest. 