\chapter{Udviklingsmiljø}\label{Udvikling}
Der er i nedenstående tabel \ref{Udvikling} opstillet de programmer og versioner der er brugt til udviklingen af Automatisk Ultralydsscanner. 

\begin{table}[htb]
\centering
\begin{tabular}{| l | p{0.20\textwidth}| }
\hline
\textbf{Emne} & \textbf{Version} \\\hline
Windows Education & 10.0.14393 \\\hline
Microsoft Visual Studio Community & 2015 \\\hline
.NET Framework & 4.5.2 \\\hline
JetBrains ReSharper Ultimate & 2016.2.2\\\hline
MeshLab & 1.3.3 \\\hline
Notepad++ & 7.2.1 \\\hline
Unity3D & 5.4.1 \\\hline
Google SketchUp & 16.1 \\\hline
MathCad & 14 \\\hline
Microsoft Kinect API & 1.8 \\\hline
UR10 & 3.1.18024 \\\hline
GitHub & 2016\\\hline
SourceTree & 1.8.3.0 \\\hline

\end{tabular}
\caption{Udviklingsmiljø}
\label{Udvikling}
\end{table}
Til operativsystem er der anvendt Windows, da .NET-frameworket og Microsoft's Kinect API virker natively på dette OS.

I projektet blev Visual Studio og udviklingsframeworket .NET (i sproget C\#) anvendt til at udvikle PC Applikation. For at beregne code coverage er der anvendt ReSharper.

Til debugging af 3D scanninger er MeshLab og Notepad++ anvendt. 

Google SketchUp er et 3D moduleringsprogam, som i projektet er blevet anvendt til at visualisere de beregninger der skulle til for at rotere Robotarm.

Til 3D visualisering af Automatisk Ultralydsscanner blev Unity anvendt. Dette gjorde det hurtigere at få at overblik over 3D kameras position-offset ift. Robotarm.

Til matematiske beregninger af Robotarms rotation blev Mathcad anvendt.

Til versionsstyring er der brugt GitHub som repository-hosting og SourceTree til git-grænseflade.

